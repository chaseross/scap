\documentclass[12pt]{article}

%\documentclass[justified, nobib]{tufte-handout2}


\usepackage{longtable}
\usepackage{natbib}           % call natbib
\setcitestyle{authoryear}     % set citation style to authoryear
\bibliographystyle{plainnat}  % use the plainnat instead of plain
\usepackage{booktabs}
\usepackage{amssymb, amsmath, amsfonts}
\usepackage{outlines}
\usepackage{fancyhdr}
\usepackage{lineno}
\usepackage{hyperref}
\pagestyle{fancy}
\usepackage{outlines}
\usepackage{caption}
\captionsetup[table]{name=Figure}


\usepackage{enumitem}
\setlist[enumerate,2]{label=\roman*)}
\setlist[enumerate,3]{label=\Roman*)}
\setlist[enumerate,4]{label=\roman*)}

%\renewcommand{\familydefault}{\sfdefault}
% ------------------------------- use \citep{FAQs} to cite
\usepackage{filecontents}
\begin{filecontents}{\jobname.bib}
@article{FAQs,
   title     = {{Frequently Asked Questions on Supervisory Capital Assessment Program}},
   publisher = "Board of Governors of the Federal Reserve System",
   author    = {{Federal Reserve}},
   year      = "2009",
   doi       = " ",
   volume    = " ",
   journal   = " ",
   issn      = " ",
   url        = {http://www.federalreserve.gov/newsevents/press/bcreg/bcreg20090225a1.pdf},
}

@article{Hirtle,
   title     = {{Macroprudential Supervision of Financial Institutions: Lessons from the SCAP}},
   publisher = "Federal Reserve Bank of New York",
   author    = "Hirtle, A. and Schuermann T. and Stiroh, K.",
   year      = "2009",
   doi       = " ",
   volume    = " ",
   journal   = "Staff Report No. 409",
   issn      = " ",
   url        = {http://www.federalreserve.gov/newsevents/press/bcreg/bcreg20090225a1.pdf},
}

@article{Design,
   title     = {{The Supervisory Capital Assessment Program: Design and Implementation}},
   publisher = "Board of Governors of the Federal Reserve System",
   author    = {{Federal Reserve}},
   year      = "2009",
   doi       = " ",
   volume    = " ",
   journal   = " ",
   issn      = " ",
   url        = {http://www.federalreserve.gov/bankinforeg/bcreg20090424a1.pdf},
}

@article{Results,
   title     = {{The Supervisory Capital Assessment Program: Overview of Results}},
   publisher = "Board of Governors of the Federal Reserve System",
   author    = {{Federal Reserve}},
   year      = "2009",
   doi       = " ",
   volume    = " ",
   journal   = " ",
   issn      = " ",
   url        = {http://www.federalreserve.gov/newsevents/press/bcreg/bcreg20090507a1.pdf},
}

@article{Term,
   title     = {{Term Sheet for Capital Assistance Program}},
   publisher = "United States Treasury",
   author    = {{U.S. Treasury}},
   year      = "2009",
   doi       = " ",
   volume    = " ",
   journal   = " ",
   issn      = " ",
   url        = {http://www.treasury.gov/press-center/press-releases/Documents/tg40_captermsheet.pdf},
}

@article{NYTimes,
   title     = {{Government Offers Details of Bank Stress Tests}},
   publisher = "New York Times",
   author    = "Andrews, E. and Dash, E.",
   year      = "2009",
   doi       = " ",
   volume    = " ",
   journal   = " ",
   issn      = " ",
   url        = {http://www.nytimes.com/2009/02/26/business/economy/26banks.html?_r=1},
}

@article{WSJ,
   title     = {{Banks Won Concessions on Tests}},
   publisher = "Wall Street Journal",
   author    = "Enrich, D. and Fitzpatrick, D. and Eckblad, M.",
   year      = "2009",
   doi       = " ",
   volume    = " ",
   journal   = " ",
   issn      = " ",
   url        = {http://www.wsj.com/articles/SB124182311010302297},
}

@article{OFS,
   title     = {{Troubled Asset Relief Program: Two Year Retrospective}},
   publisher = "Office of Financial Stability",
   author    = {{Office of Financial Stability}},
   year      = "2010",
   doi       = " ",
   volume    = " ",
   journal   = " ",
   issn      = " ",
   url        = {https://www.treasury.gov/press-center/news/Documents/TARP\%20Two\%20Year\%20Retrospective_10\%2005\%2010_transmittal\%20letter.pdf},
}

@book{Bernanke,
   title     = {{The Courage to Act}},
   publisher = "W.W. Norton \& Company",
   author    ={Ben Bernanke},
   year      = "2015",
   doi       = " ",
   volume    = " ",
   journal   = " ",
   issn      = "039324721X",

}

@article{CFR,
   title     = {{12 CFR part 225, Appendix A, Section II.A.I.}},
   publisher = "GAO",
   author    ={CFR},
   year      = " ",
   doi       = " ",
   volume    = " ",
   journal   = " ",
   issn      = " ",
   url 	= {https://www.law.cornell.edu/cfr/text/12/part-225/appendix-A},
}



\end{filecontents}
% -------------------------------




\begin{document}

%% DELETE BELOW TO GO BACK TO DEFAULT %%%%%%
\lhead{ }
%\rhead{\small Supervisory Capital Assessment Program}
%\rfoot{\small \thepage}

\renewcommand{\headrulewidth}{0.0pt}
\renewcommand{\footrulewidth}{0.0pt}

%% DELETE ABOVE TO GO BACK TO DEFAULT %%%%%%

\title{Supervisory Capital Assessment Program}%\thanks{This case is one of a series of intervention studies.}}
\author{Chase Ross\thanks{ Project Editor, Yale Program on Financial Stability, Yale School of Management \newline \texttt{\href{mailto:chase.ross@yale.edu}{chase.ross@yale.edu}}}, Andrew Metrick\thanks{Michael H. Jordan Professor of Finance and Management, and Yale Program on Financial Stability Program Director, Yale School of Management \newline \texttt{\href{mailto:andrew.metrick@yale.edu}{andrew.metrick@yale.edu}}}}
\date{December 15, 2015}



\maketitle

\begin{abstract}
%\textbf{Abstract}
%\newline
%\newline
When President Obama took office in 2009, the Treasury focused on
restarting bank lending and repairing the ability of the banking system
as a whole to perform the role of credit intermediation. In order to do
so, the Treasury needed to raise public confidence that banks had
sufficient buffers to withstand even a very adverse economic scenario,
especially given heightened uncertainty surrounding the outlook for the
U.S. economy and potential losses in the banking system. The Supervisory
Capital Assessment Program (SCAP)---the so-called ``stress
test''---sought to rigorously measure the resilience of the largest bank
holding companies. Those found to have insufficient capital buffers were
able to raise funds from the private sector, and if unable to do so, the
Capital Assistance Program (CAP) would capitalize the firm with public
capital. Ultimately, the SCAP test results were accepted as credible;
all except one of the tested firms raised more than enough private capital in
the following six months to fill the capital shortfall calculated by the
the SCAP and the CAP was unused.
\newline
\newline
%\textbf{JEL Classification}: G01, G28, H12, H81
%\newline
\textbf{Keywords}: stress test, public capital injection, tier 1 capital, crisis intervention

\end{abstract}
\newpage
\tableofcontents
\newpage

\section{Overview}

\subsection{Background}

In February 2009, the Treasury announced the Capital Assistance Program
(CAP) in conjunction with the Supervisory Capital Assessment Program
(SCAP).\footnote{To learn more about the SCAP and CAP, click here: \url{http://som.yale.edu/ypfs}} Their purpose was ``to restore confidence throughout the
financial system that the nation's largest banking institutions have a
sufficient capital cushion against larger than expected future losses,
should they occur due to a more severe economic environment, and to
support lending to creditworthy borrowers.'' The Treasury was concerned that the lack of near-term private capital inflows was large enough to
break the dynamic between the perceived shortage of capital in the
system and the loss of confidence that shortage engendered in the health
of individual institutions in the strained economic environment.
\citep{Term}.

\subsection{Program Description}

To restore confidence in the banking system, the SCAP employed a stress test and publicly
released the results and methodology of the test. The stress test measured whether banks had enough capital to withstand a hypothetical adverse economic event by examining each firm's Tier 1 Common capital and Tier 1 capital ratios, with targets of four and six percent respectively. Firms that required
additional capital to meet their SCAP buffers were given six months to
raise private capital and the firms unable to do so could then rely upon
public capital, sourced from TARP funds via the CAP. Smaller firms not
included in the SCAP were eligible to apply for Treasury funds through
the CAP if they experienced difficulty raising adequate capital.

U.S. bank holding companies (BHCs) with assets in excess of \$100
billion on a consolidated basis\footnote{As measured according to the
  firms' assets report for 2008Q4 in the Federal Reserve's Consolidated
  Financial Statements for Bank Holding Companies (FR Y-9C).} were
required to participate in the
SCAP, and had access to the CAP immediately as a means to build any
necessary additional buffer (although they could also raise private capital
over the following 6 months). The Treasury worked with the Board of Governors of the Federal Reserve System, the Federal Reserve Reserve Banks, the Federal Deposit Insurance Corporation and the Office of the Comptroller of the Currency to develop this stress test to determine the health of the
relevant financial institutions. As a result, the 19 largest BHCs, which
together held two-thirds of assets and more than one-half of the loans
in the U.S. banking system, participated in the SCAP on a mandatory
basis.\citep{OFS}

While the SCAP was similar to stress tests that firms undertook as part
of their ongoing risk management, the objective of this program was ``to
conduct a comprehensive and consistent assessment simultaneously''
across the largest BHCs using a common set of macroeconomic scenarios
and a common forward-looking conceptual framework. This framework
allowed supervisors to apply a consistent and systematic approach across
firms to evaluate projected losses and revenue estimates submitted by
the firms and to conduct cross-firm analysis on the aggregate. The SCAP
was considerably more comprehensive than stress tests that focus on
individual business lines because it simultaneously incorporated all
major assets and revenue sources for each firm.\citep{Design}

The SCAP involved the projection of losses on loans, assets held in
investment portfolios, and trading-related exposures, as well as a
firm's capacity to absorb losses that all combined to determine a
sufficient capital level to support lending under a worse-than-expected
macroeconomic scenario. Given the heightened uncertainty about the
economic outlook and losses in the banking system, and the possibility
for adverse economic outcomes to be magnified through the banking
system, supervisors believed it ``prudent for large BHCs to hold
substantial capital to absorb losses'' should the economic downturn be
longer and deeper than anticipated.\citep{Design} In this sense, the
test represented a set of ``what-if'' scenarios and were not a set of
estimates or projections.\citep{Results}

The SCAP was an important complement to the Treasury's support of the
banking system. The Federal Reserve System (Fed) believed that the SCAP
would help ensure the strength of the banking sector and revive markets'
confidence in the banks. It believed that the program would help protect
the taxpayers' investments in U.S. financial institutions.\citep{Results} It also believed the high level of transparency incorporated into the SCAP would help
rebuild confidence in the banking system. In its results, the Fed noted,
``{[}t{]}he decision to depart from the standard practice of keeping
examination information confidential stemmed from the belief that
greater clarity around the SCAP process and findings will make the
exercise more effective at reducing uncertainty and restoring confidence
in our financial institutions.'' Some two weeks later, Treasury
Secretary Geithner stated before the Senate Banking Committee on
May 20, 2009 that the review conducted under the SCAP was indeed
``helping to increase confidence in the financial system.''

\subsection{Outcomes}

The Fed provided loan loss rates under two different macroeconomic
environments ---``baseline'' and ``more adverse'' --- to guide SCAP
estimates. The SCAP found a ``more adverse'' economic scenario would result
in losses of some \$600 billion among the 19 largest BHCs from 2009 to
2010, implying total losses of \$950 billion from mid-2007 to the end of
2010 when charge-offs and write-downs of
about \$350 billion already recognized by the firms were included. The same firms held some \$835
billion in Tier 1 capital in Q4 2008 and all exceeded their minimum
regulatory capital standards.\citep{Results}

This suggested that the firms had enough capacity to handle the \$600 billion
in losses through 2010, and the firms would also generate revenues
through the same period that could be used to offset losses, although
firms would also need to build reserves against credit problems beyond
2010. Combining these factors, supervisors found that the firms needed to add
\$75 billion more in capital buffers to meet the SCAP target buffers by
the end of 2010 after accounting for Q1 2009 revenues under the ``more
adverse'' scenario.

Nine of the 19 firms involved were sufficiently capitalized for the four
percent Tier 1 Common capital and six percent Tier 1 capital targets
established by the SCAP and almost all the firms had enough overall Tier
1 capital to withstand the ``more adverse'' scenario. Ten firms required
\$75 billion in additional capital. These ten firms had virtually no
shortfall in overall Tier 1 capital\footnote{Tier 1 capital is ``the sum
  of core capital elements less any amount of goodwill, other intangible
  assets, interest-only strips receivables, deferred tax assets,
  nonfinancial equity investments, and other items.'' (12 CFR part 225,
  Appendix A).} but instead had insufficient common equity. Therefore,
the ten firms needed to increase their capital by increasing their Tier 1
Common capital\footnote{Tier 1 Common capital is ``Tier 1 capital less
  the non-common elements of tier 1 capital, including perpetual
  preferred stock and related surplus, minority interest in
  subsidiaries, trust preferred securities and mandatory convertible
  preferred securities.'' (12 CFR \S 225.8). It reflects the fact that
  common equity is the first element of the capital structure to absorb
  losses, offering protection to more senior parts of the capital
  structure and lowering the risk of insolvency.~~All else equal, more
  Tier 1 Common capital gives a BHC greater permanent loss absorption
  capacity and a greater ability to conserve resources under stress by
  changing the amount and timing of dividends and other distributions.\citep{Results}}. Figure \ref{figure1} shows the additional capital requirements by firm.

Initial reactions to the stress test were ``generally positive,''
although some market commentators questioned the use of the Tier 1
Common capital ratio instead of the tangible common equity (TCE) ratio,
which many had originally assumed would be the supervisors' benchmark.
Where the Tier 1 Common capital ratio is the capital of common
shareholders as a percent of risk-weighted assets, the TCE ratio is
equity less intangible assets, goodwill and preferred stock equity as a
percent of tangible assets, which are the company's total assets less
goodwill and intangibles.\citep{WSJ}

The Office of Financial Stability (OFS) reported that since the release
of the results of the SCAP, the institutions subject to the stress test
altogether increased their requisite capital by over \$150 billion
through September 2010. Additionally, the OFS emphasized that after the SCAP --
participating BHCs raised this additional capital, more than 80 other
banks raised sufficient capital to repay the TARP investments made by
the Treasury (although not through the CAP).\citep{OFS}

\begin{table}[htbp]
\setlength\LTleft\fill
\setlength\LTright{0pt}
\begin{longtable}[l]{@{\extracolsep{\fill}}@{}ll@{}ll@{}}
\caption{SCAP Results}\label{figure1}\\
\toprule
\textbf{Firm} & \textbf{Additional Capital Required} &\tabularnewline
\midrule
\endhead
Bank of America & \$33.9 billion & ~\tabularnewline
Wells Fargo & \$13.7 billion &\tabularnewline
GMAC & \$11.5 billion & ~\tabularnewline
Citigroup & \$5.5 billion &\tabularnewline
Regions Financial Corp. & \$2.5 billion & \tabularnewline
SunTrust Banks & \$2.2 billion &\tabularnewline
Morgan Stanley & \$1.8 billion & \tabularnewline
KeyCorp & \$1.8 billion &\tabularnewline
Fifth Third Bank & \$1.1 billion & \tabularnewline
PNC & \$0.6 billion &\tabularnewline
American Express & Adequate & \tabularnewline
Bank of New York Mellon & Adequate &\tabularnewline
BB\&T & Adequate & ~\tabularnewline
Capital One & Adequate &\tabularnewline
Goldman Sachs & Adequate & \tabularnewline
JP Morgan Chase & Adequate &\tabularnewline
MetLife & Adequate & ~\tabularnewline
State Street & Adequate &\tabularnewline
U.S. Bancorp & Adequate & \tabularnewline
\bottomrule
\multicolumn{3}{l}{\footnotesize Source: Federal Reserve (May 7, 2009).}
\end{longtable}
\end{table}

\section{Key Design Decisions}

\subsection{Supervisors published the design of the stress test and
its results at a very detailed
level.}

A key feature of the SCAP was the level of transparency involved. Under
the SCAP, supervisors published the design of the test and its
results at a detailed level, much of which was usually confidential and
not released publicly. The Treasury's Office of Financial Stability
described this transparency as a novel step taken due to the
unprecedented need to restore confidence. The Fed also noted that this
was an unprecedented exercise, and the extraordinary economic and
financial conditions that precipitated the program led the supervisors
to take the unusual step of publicly reporting the findings of this
supervisory exercise. The transparency and process sought to allow
markets to credibly assess the overall health of the BHCs as well
as to make direct comparisons across the firms. Figure \ref{figure2} provides a sample
of the information published at the firm-specific level.

\begin{table}[htbp]
\setlength\LTleft\fill
\setlength\LTright{0pt}
\begin{longtable}[l]{@{\extracolsep{\fill}}@{}ll@{}rl@{}}
\caption{Estimates for Bank of America Corporation for the More Adverse Economic Scenario} \label{figure2}\\
\toprule
\textbf{At December 31, 2008} & \$ Billion  & As \% of RWA \tabularnewline
\midrule
\endhead
Tier 1 Capital & 173.2 & 10.6 \tabularnewline
\hspace{1em} Tier 1 Common Capital & 74.5 & 4.6\tabularnewline
Risk-Weighted Assets & 1,633.8 &~\tabularnewline
~ & ~ & ~\tabularnewline
\textbf{More Adverse Scenario} & ~ & ~\tabularnewline
\midrule
\textbf{Estimated for 2009 and 2010} & \$ Billion & \,\, As \% of Loans \tabularnewline
\midrule
Total Estimated Losses\textsuperscript{1}& 136.6 & ~\tabularnewline
\hspace{1em} First Lien Mortgage& 22.1 & 6.8 \tabularnewline
\hspace{1em} Second/Junior Lien Mortgage&21.4 & 13.5 \tabularnewline
\hspace{1em} Commercial and Industrial Loans& 15.7 & 13.5\tabularnewline
\hspace{1em} Commercial Real Estate Loans & 9.4 & 9.1\tabularnewline
\hspace{1em} Credit Card Loans& 19.1 & 23.5\tabularnewline
\hspace{1em} Securities (AFS and HTM)& 8.5 & -na-\tabularnewline
\hspace{1em} Trading \& Counterparty& 24.1 & -na-\tabularnewline
\hspace{1em} Other\textsuperscript{2}& 16.4 & -na-\tabularnewline
\hspace{1em} Memo: Purchase Accounting Adjustment&13.3 & ~\tabularnewline
Resources Other Than Capital to Absorb Losses\textsuperscript{3}& 74.5 & ~\tabularnewline

~ & ~ & ~\tabularnewline
\textbf{SCAP Buffer Added for More Adverse Scenario\textsuperscript{4}} & ~ & ~\tabularnewline
Indicated SCAP Buffer as of December 31, 2008 & 46.5 & ~\tabularnewline
\hspace{1em} Less: Capital Actions and Effects of Q1 2009 Results\textsuperscript{5}& 10.9 & ~\tabularnewline
\hspace{2em} Other Capital Actions\textsuperscript{6}& 1.8 & ~\tabularnewline
SCAP Buffer & 33.9 &~\tabularnewline
\bottomrule
\multicolumn{3}{l}{\footnotesize\textsuperscript{1} Before purchase accounting adjustment }\tabularnewline
\multicolumn{3}{l}{\footnotesize\textsuperscript{2} Includes other consumer and non-consumer loans and misc. commitments and obligations}\tabularnewline
\multicolumn{3}{l}{\footnotesize\textsuperscript{3} Resources to absorb losses include PPNR less change in ALLL}\tabularnewline
\multicolumn{3}{l}{\footnotesize\textsuperscript{4} SCAP buffer is defined as additional Tier 1 Common/contingent Common}\tabularnewline
\multicolumn{3}{l}{\footnotesize\textsuperscript{5} Capital actions include completed or contracted transactions since Q4 2008}\tabularnewline
\multicolumn{3}{l}{\footnotesize\textsuperscript{6} Capital benefit from risk-weighted assets impact of eligible asset guarantee}\tabularnewline
\multicolumn{3}{l}{\footnotesize Note: Numbers may not sum due to rounding}\tabularnewline
\multicolumn{3}{l}{\footnotesize Source: Federal Reserve (May 7, 2009)}\tabularnewline
\end{longtable}

\end{table}

There were many critics of the transparency of the SCAP, with the criticism
falling broadly into three groups. First, some felt the transparency would
violate the precedent of ``supervisor confidentiality'' that helped allay
banks' reticence in sharing proprietary information with their
supervisors. ``In normal times, assurances of confidentially increase
banks' willingness to cooperate with examiners by allaying any concern
that their proprietary information would be obtained by competitors.''\citep{Bernanke}

A second criticism was the possibility that full disclosures of a bank's
capital position would reveal a firm as weaker than the market
originally thought. In this case, the bank could suffer new runs and the
market's confidence could worsen. Former Chairman Bernanke noted:

\begin{quote}
In the atmosphere of fear and uncertainty that prevailed in early 2009,
we could not dismiss the possibility that disclosing banks' weakness
could further erode confidence, possibly leading to new runs and further
sharp declines in bank stock prices. Fed Board members agreed, however,
that releasing as much information as possible was the best way to
reduce the paralyzing uncertainty about banks' financial health.
\citep{Bernanke}
\end{quote}

A third criticism focused on the supervisor's likelihood to whitewash
the results of the test in order to protect the weaker banks, especially given the public disclosure of the test results. In
February 2009, after the Federal Reserve announced the first details on
how it intended to test the banks, the markets and media felt the test's
results would likely be overly optimistic. The day of the announcement,
the \emph{New York Times} noted that ``\ldots{}analysts say
the administration's worst projections, which it describes as unlikely,
are not much more dire than what many private forecasters already
expect.'' The \emph{Times} quoted a bank analyst: ``It sure sounds to me
like they are designing this to make it sound like the banking system is
in great shape.''\citep{NYTimes}

\subsection{The Fed helped increase credibility of the test by
considering two different macroeconomic scenarios, with the ``more adverse''
scenario projecting more severe loss estimates than many market
analysts.}

The Fed provided indicative loss rates under two different macroeconomic
environments---``baseline'' and ``more adverse''---to guide BHCs'
estimates. The Fed supervisors chose particularly severe conditions in
the ``more adverse'' scenario, with the estimated commercial bank
two-year loan loss rate at 9.1\%, a rate higher than any observed from
1920 to 2008. The scenario also included estimates of unemployment at
its highest level since the 1930s. Figure \ref{figure3} details the two scenarios.
Firms used these two scenarios to conduct their own estimates, and then
the Fed staff analyzed each firm's estimates independently.\citep{Results}


\begin{table}[htbp]
\setlength\LTleft\fill
\setlength\LTright{0pt}
\begin{longtable}[l]{@{\extracolsep{\fill}}@{}ll@{}ll@{}}
\caption{SCAP Economic Scenario}\label{figure3}\\
\toprule
~ & 2009 & 2010 \tabularnewline
\midrule
\endhead
\textbf{Real GDP}\textsuperscript{1}  & ~ & ~\tabularnewline
SCAP Average Baseline\textsuperscript{2} & -2.0 & 2.1 \tabularnewline
Consensus Forecasts & -2.1 & 2.0\tabularnewline
Blue Chip & -1.9 &2.1 \tabularnewline
Survey of Professional Forecasters & -2.0 & 2.2 \tabularnewline
SCAP Alternative More Adverse & -3.3 & 0.5\tabularnewline

~ & ~ & ~\tabularnewline
\textbf{Civilian Unemployment Rate}\textsuperscript{3}  & ~ & ~\tabularnewline
SCAP Average Baseline & 8.4 & 8.8 \tabularnewline
Consensus Forecasts & 8.4 & 9.0\tabularnewline
Blue Chip & 8.3 & 8.7 \tabularnewline
Survey of Professional Forecasters & 8.4 & 8.8 \tabularnewline
SCAP Alternative More Adverse & 8.9 & 10.3\tabularnewline

~ & ~ & ~\tabularnewline
\textbf{House Prices}\textsuperscript{4}  & ~ & ~\tabularnewline
SCAP Average Baseline & -14 & -4 \tabularnewline
SCAP Alternative More Adverse & -22 & -7\tabularnewline
\bottomrule
\multicolumn{3}{l}{\footnotesize \textsuperscript{1} Percent change in annual average.} \tabularnewline
\multicolumn{3}{l}{\footnotesize \textsuperscript{2} Baseline forecasts equal the average projections in February 2009.} \tabularnewline
\multicolumn{3}{l}{\footnotesize \textsuperscript{3} Annual average.} \tabularnewline
\multicolumn{3}{l}{\footnotesize \textsuperscript{4} Case-Shiller 10-City Composite, percent change, Q4/Q4.} \tabularnewline
\multicolumn{3}{l}{\footnotesize Source: Federal Reserve (May 7, 2009).} \tabularnewline

\end{longtable}

\end{table}

\subsection{The stress test was based on targets for the amount of
Tier 1 capital firms should
hold.}

Fed staff examined both the level of capital in each BHC as well as the
composition of the capital held by evaluating the Tier 1 risk-based
capital ratio, which is the ratio of Tier 1 capital to risk-weighted
assets. The Fed also evaluated common equity's share of Tier 1 capital.
Tier 1 capital was required to be at least half of qualifying total
capital. In practice, these measures are referred to as the Tier 1
Common capital ratio and the Tier 1 capital ratio where the former
excludes preferred shares and non-controlling interests and the latter
includes them. Regulators focused on Tier 1 Common capital specifically as it is
the first to absorb losses and so acts as a permanent cushion that BHCs
can adjust to changing circumstances by varying dividends and their
timing.\citep{CFR}

In designing and implementing the test, the Fed sought to answer two
questions to determine the resources required to withstand a ``more
adverse'' macroeconomic environment:

\begin{enumerate}
\def\labelenumi{\arabic{enumi}.}
\item
  If the economy follows the ``more adverse'' scenario, how much
  additional Tier 1 capital would an institution need today to be able
  to have a Tier 1 risk-based ratio in excess of six percent at year-end
  2010?
\item
  If the economy follows the ``more adverse'' scenario, how much
  additional Tier 1 Common capital would an institution need to have
  today to have a Tier 1 Common capital risk-based ratio in excess of
  four percent at year-end 2010?
\end{enumerate}

The test design also assumed that the institutions would continue to
operate under the regulatory framework existing as of December 31, 2008
and under any significant changes in the framework that would take place
over the next two years. The stress test included an assessment of
capital at the end of 2010 capturing expected losses in 2011, as BHCs
participating in the SCAP booked the majority of their assets on an
accrual basis.\citep{Design}

\subsection{The Fed applied ``other than temporary impairment'' charges to
securities that a bank was likely to sell or be forced to
sell.}

Particular focus centered on losses in available-for-sale (AFS) and
held-to-maturity (HTM) portfolios, as well as counterparty credit risk.
Losses in these items would ultimately represent the largest share of
total losses. Supervisors analyzed whether securitized assets would
become impaired over their lifetime. If this were the case, and if
credit support was insufficient to cover expected losses, the security
was written-down to fair value.

In its analysis, the Fed considered whether firms intended to sell a
security or ``whether it is more-likely-than-not that firms will be
required to sell the security before recovery of its cost basis.'' Both
scenarios triggered an ``other than temporary impairment'' charge.
Regulators felt it best practice, following the Financial Accounting
Standards Board's guidance, to account for the eventuality that firms
might not be able to hold a security to recovery under more stressful
conditions. For securities determined to be other than temporarily
impaired, the loss was the difference between the amortized cost basis
and fair value.\citep{Results}

\subsection{Q1 2009 revenues were included in the stress test
calculation.}

Market participants also questioned whether Q1 2009 results would be
included in the estimates. After a period, regulators decided to include
these revenues. In the subsequently published information for each
institution, a line item addressed this explicitly as in footnote 3 in
Figure 2 noting, ``{[}c{]}apital actions include completed or contracted
transactions since Q4
2008.''

\subsection{BHCs with capital shortfalls revealed by the SCAP would
ultimately have access to U.S. Treasury funds if the firm was unable to
raise private
capital.}

If the SCAP test indicated a bank needed more capital, the Treasury
would provide the additional capital buffer through the CAP in the form
of convertible preferred securities should the bank be unable to raise
enough capital through private sources. These convertible preferred
holdings would be convertible to common equity ``if needed to retain the
confidence of investors or to meet supervisory expectations regarding
the amount and composition of capital.'' \citep{Term} The
Treasury offered the CAP to all banks and qualifying financial
institutions, unlike the SCAP which was limited to the
19 largest BHCs. BHCs had six months to raise additional capital after the publication of the SCAP
results. A firm could, however, apply to the CAP immediately after the release of the SCAP results, but delay actual CAP funding for the six months while it
raised as much private capital as possible.

Qualifying financial institutions (QFIs) were eligible to apply to the
CAP. QFIs included BHCs, financial holding companies, insured depository
institutions, and savings and loan holding companies, that were
organized and operating in the United States, and deemed viable by the
appropriate federal banking agency. Financial institutions controlled by
foreign entities were ineligible.\citep{Term}

\subsection{Regulators conducted the stress test on BHCs with more
than \$100 billion in assets on a consolidated
basis.}

A government report highlighted the moral hazard concerns relating to
the stress test under the SCAP. It pointed out that some saw the focus
of SCAP on BHCs with more than \$100 billion in asset size as a formal
demarcation of ``too-big-to-fail.'' The report also expressed that this
focus of the SCAP created an impression that the federal government
would protect the 19 BHCs at least for the duration of the financial
crisis due to their size.\citep{OFS}

This concern somewhat materialized in rating agencies' practice of
upgrading the largest banks based on their access to extensive support
from the federal government in 2009. For example, Moody's upgraded
ratings for deposits and senior debt issued by the six largest U.S.
banks, based on its expectation of ``a very high probability of systemic
support'' for such banks from the U.S. government. Moreover, Standard \&
Poor's gave a rating upgrade to Citigroup and the agency added that it
would have been rated four notches lower with no government assistance.
On the other hand, Standard \& Poor's downgraded a Citibank subsidiary,
Citibank Korea Inc., in its stand-alone rating because it felt that
there was ``uncertainty'' about whether the U.S. government wanted
Citigroup providing additional support to noncore oversea affiliates.\citep{OFS}

\section{Evaluation}

The SCAP was widely seen as a turning point for the U.S. banking system,
due both to the SCAP's credibly tough loss estimates, as well as the
capital backstop provided by the CAP. In order to make the test credible,
and therefore increase the likelihood the market accepted its results,
loss estimates were particularly severe---higher even than loss rates
during the Great Depression. Moreover, the availability of public funds
via the CAP ``gave regulators the right incentives.''\citep{Bernanke}
Without the CAP backstop, regulators may have been inclined to present
over-optimistic assessments of the weakest banks. The SCAP's results ultimately proved valuable information for market participants and increased confidence in the banking system. As a result, ``BHCs responded with substantial actions to improve capital [positions].''\citep{Hirtle}
%\section{Resources Cited}
%
%\begin{itemize}
%\item
%  \href{http://www.federalreserve.gov/newsevents/press/bcreg/bcreg20090225a1.pdf}{\emph{FAQs
%  on Supervisory Capital Assessment Program}}. Board of Governors of the
%  Federal Reserve System, February 25, 2009. \emph{Federal Reserve
%  document outlining initial design details and purpose of the SCAP}
%\item
%  \href{http://www.federalreserve.gov/bankinforeg/bcreg20090424a1.pdf}{\emph{The
%  Supervisory Capital Assessment Program: Design and Implementation}},
%  Board of Governors of the Federal Reserve System, April 24, 2009.
%  \emph{Federal Reserve document outlining design details of the SCAP}
%\item
%  \href{http://www.federalreserve.gov/newsevents/press/bcreg/bcreg20090507a1.pdfhttp:/www.federalreserve.gov/newsevents/press/bcreg/bcreg20090507a1.pdf?bcsi_scan_D92198957E035F0B=0\&bcsi_scan_filename=bcreg20090507a1.pdf}{\emph{The
%  Supervisory Capital Assessment Program: Overview of Results}}, Board
%  of Governors of the Federal Reserve System, May 7, 2009. \emph{Federal
%  Reserve document that announced the results of the SCAP}
%\item
%  \href{http://www.treasury.gov/press-center/press-releases/Documents/tg40_captermsheet.pdf}{\emph{Term
%  Sheet for Capital Assistance Program}}, U.S. Treasury. \emph{Treasury
%  document discussing terms of investments made via the CAP}
%\item
%  \href{http://www.nytimes.com/2009/02/26/business/economy/26banks.html}{\emph{Government
%  Offers Details of Bank Stress Tests}}, New York Times, February 25,
%  2009. \emph{Article discussing immediate reception to stress tests'
%  proposed methodology}
%\item
%  \href{http://www.wsj.com/articles/SB124182311010302297}{\emph{Banks
%  Won Concessions on Tests}}, Wall Street Journal, May 9, 2009.
%  \emph{Article discussing the implications for targeting tangible
%  common equity compared to common equity Tier 1}
%\item
%  \href{http://www.treasury.gov/press-center/news/Documents/TARP\%20Two\%20Year\%20Retrospective_10\%2005\%2010_transmittal\%20letter.pdf}{\emph{Troubled
%  Asset Relief Program: Two Year Retrospective}}, Office of Financial
%  Stability. \emph{Office of Financial Stability report discussing the
%  program and its outcomes in the context of the wider swath of TARP}
%\item
%  \href{https://www.treasury.gov/press-center/press-releases/Pages/tg139.aspx.}{\emph{Statement
%  by Timothy F. Geithner U. S. Secretary of the Treasury before the
%  Senate Banking Committee}}, May 20, 2009. \emph{Secretary Geithner
%  discusses the initial impact and market response of the SCAP's
%  results}
%\item
%  \emph{The Courage to Act}, Bernanke, 2015. \emph{Former Chairman
%  Bernanke's memoir, which includes passages on policymakers'
%  considerations when designing and implementing SCAP}
%\end{itemize}

\phantomsection

\addcontentsline{toc}{section}{4. References}

\nocite{*}
\bibliography{\jobname}

\section{Appendix A - List of Resources}

\subsection{Summary of Program}

\begin{itemize}
\item
\emph{FAQs
  on Supervisory Capital Assessment Program}, Board of Governors of the
  Federal Reserve System, February 25, 2009. \emph{Federal Reserve
  document outlining initial design details and purpose of the SCAP.}\url{http://www.federalreserve.gov/newsevents/press/bcreg/bcreg20090225a1.pdf}
\item
   \emph{The Supervisory Capital Assessment Program: Overview of Results}, Board
  of Governors of the Federal Reserve System, May 7, 2009. \emph{Federal
  Reserve document that announced the results of the SCAP.}\url{http://www.federalreserve.gov/newsevents/press/bcreg/bcreg20090507a1.pdf}
\item
\emph{FAQs
  on Capital Purchase Program Repayment and Capital Assistance
  Program}, Treasury, 2008. \emph{Treasury Document discussing
  relationship between SCAP and the CAP.}\url{http://www.treasury.gov/initiatives/financial-stability/TARP-Programs/bank-investment-programs/scap-and-cap/Documents/FAQ_CPP-CAP.pdf}
\item
\emph{Term
  Sheet for Capital Assistance Program}, U.S. Treasury. \emph{Treasury
  document discussing terms of investments made via the CAP.}\url{http://www.treasury.gov/press-center/press-releases/Documents/tg40_captermsheet.pdf}
\end{itemize}

\subsection{Implementation Documents}
\begin{itemize}
\item
\emph{The
  Supervisory Capital Assessment Program: Design and Implementation},
  Board of Governors of the Federal Reserve System, April 24, 2009.
  \emph{Federal Reserve document outlining design details of the SCAP.}\url{http://www.federalreserve.gov/bankinforeg/bcreg20090424a1.pdf}
\end{itemize}

\subsection{Legal/Regulatory Guidance}

\begin{itemize}
\item
\emph{12
  CFR part 225, Appendix A, Section II.A.I}. \emph{Provides definitions
  of Tier 1 capital ratios and describes components of qualifying
  capital.}\url{https://www.law.cornell.edu/cfr/text/12/part-225/appendix-A}
\item
\emph{Staff
  Positions 115-2 and FAS 124-2, Recognition and Presentation of
  Other-Than-Temporary Impairments}, Financial Accounting Standards
  Board (FASB), April 9, 2009. \emph{Guidance describing the suggested
  process to account for debt securities held in the AFS and HTM
  accounts and when a firm must recognize other than temporary
  impairments.}\url{http://www.gasb.org/jsp/FASB/Document_C/DocumentPage\%3Fcid=1176154545419\%26acceptedDisclaimer=true}
\end{itemize}

\subsection{Press Releases/Announcements}

\begin{itemize}
\item
\emph{Agencies
  to Begin Forward-Looking Economic Assessment}, Federal Reserve, FDIC,
  OCC, OTS Press Release, February 25, 2009. \emph{Press release
  announcing the SCAP.}\url{http://www.federalreserve.gov/newsevents/press/bcreg/20090225a.htm}
\item
\emph{Joint
  Statement by Secretary of the Treasury Timothy F. Geithner, Chairman
  of the Board of Governors of the Federal Reserve System Ben S.
  Bernanke, Chairman of the Federal Deposit Insurance Corporation Sheila
  Bair, and Comptroller of the Currency John C. Dugan on The Treasury
  Capital Assistance Program and the Supervisory Capital Assessment
  Program}, May 6, 2009. \emph{Press release the day before the SCAP's
  results were to be announced, describing how to understand the SCAP
  results and the nature of the mandatory convertible preferred capital
  to be used in the CAP.}\url{http://www.federalreserve.gov/newsevents/press/bcreg/20090506a.htm}
\item
\emph{SCAP
  Results}, Federal Reserve, May 7, 2009. \emph{Press release which
  announces the results of the SCAP.}\url{http://www.federalreserve.gov/newsevents/press/bcreg/20090507a.htm}
\item
\emph{Statement
  by Timothy F. Geithner U. S. Secretary of the Treasury before the
  Senate Banking Committee}, May 20, 2009. \emph{Secretary Geithner
  discusses the initial impact of and market response to the SCAP's
  results.}\url{https://www.treasury.gov/press-center/press-releases/Pages/tg139.aspx}
\end{itemize}

\subsection{Media Stories}

\begin{itemize}
\item
\emph{Government
  Offers Details of Bank Stress Tests}, New York Times, February 25,
  2009. \emph{Article discussing immediate reception to stress test's
  proposed methodology.}\url{http://www.nytimes.com/2009/02/26/business/economy/26banks.html}
\item
\emph{Bank
  Capital Gets Stress Test}, Wall Street Journal, February 26, 2009.
  \emph{Article discussing initial media and market reaction to the test
  plans including criticisms and discussing the possibility of
  nationalizing the banks should the test results give poor indications
  of the banks' health.}\url{http://www.wsj.com/articles/SB123557705225772665}
\item
\emph{Banks
  Need at Least \$65 billion in Capital}, Wall Street Journal, May 7,
  2009. \emph{Article discussing the initial reaction of the test
  results.}\url{http://www.wsj.com/articles/SB124163049445592523}
\item
\emph{Banks
  Won Concessions on Tests}, Wall Street Journal, May 9, 2009.
  \emph{Article discussing the implications for targeting tangible
  common equity compared to common equity Tier 1.}\url{http://www.wsj.com/articles/SB124182311010302297}
\end{itemize}

\subsection{Key Academic Papers}

\begin{itemize}
\item
\emph{Macroprudential
  Supervision of Financial Institutions: Lessons from the SCAP},
  Hirtle, Schuermann, Stiroh, 2009, NYFRB Staff Report. \emph{Paper
  discussing the SCAP's combination of micro and macro prudential
  perspectives, describing key features of the SCAP in detail as well as
  how these features could be incorporated into bank supervision in the
  future.}\url{https://www.newyorkfed.org/medialibrary/media/research/staff_reports/sr409.pdf}
\end{itemize}

\subsection{Government Reports/Assessments}

\begin{itemize}
\item
\emph{Consolidated
  Financial Statements for Bank Holding Companies: Reporting Form FR
  Y-9C}, Board of Governors of the Federal Reserve System, December
  2008. \emph{Document that determined whether a BHC held sufficient
  assets on a consolidated basis to be include in the SCAP.}\url{http://www.federalreserve.gov/reportforms/forms/FR_Y-9C20081231_f.pdf}
\item
\emph{Troubled
  Asset Relief Program: Two Year Retrospective}, Office of Financial
  Stability, October 2010. \emph{Office of Financial Stability report
  discussing the program and its outcomes in the context of the wider
  swath of TARP.}\url{http://www.treasury.gov/press-center/news/Documents/TARP\%20Two\%20Year\%20Retrospective_10\%2005\%2010_transmittal\%20letter.pdf}
\end{itemize}

\section{Appendix B - Road Map}

The following is a list of the key design decisions that will likely have to be made in implementing a program similar to the Supervisory Capital Assessment Program (SCAP), a ``stress test'' program intended to assess the capital needs of large bank holding companies (BHCs) during a period of heightened uncertainty around potential losses in the banking system.

\subsection{Key Questions}

\begin{outline}[enumerate]

\1 Which agency or agencies have the authority and expertise to conduct the stress test?
\2 What is the basis of this authority?
\2 Does the agency conducting the test collect the necessary information as part of its regular bank examination process, or will the agency require collection of more granular data for the test?
\2 What particular elements/terms must be satisfied to fit within the authority?
\2 After designing, have all required elements been satisfied?
\1 What, if any, capital backstop should be available to firms which fail the stress test?
\2 Is any additional authority required in order to provide a capital backstop?
\2 How can the backstop be structured to compel firms to first raise private capital and use the public capital as a less preferred option?
\2 How long should firms be allowed to seek private capital before turning to the public backstop?
\2 SCAP: The Capital Assistance Program (CAP) was the public backstop for the SCAP tested firms. Firms were given 6 months to raise private capital.
\1 Which firms are included in the stress test?
\2 How many firms can be credible tested given the testing agency's resources?
\2 SCAP: BHCs with assets greater than \$100 billion, as reported in the 2008Q4 in the Federal Reserve’s Consolidated Financial Statements for Bank Holding Companies (FR Y-9C). As a result, the 19 largest BHCs, which together held two-thirds of assets and more than one-half of the loans in the U.S. banking system, participated in the SCAP on a mandatory basis.
\1 How transparent should the test results be? What level of granularity for estimates should be publicly available?
\1 How can the regulators ensure the test is viewed as credible?
\2 What metric or measure should regulators target to assess capital adequacy?
\3 Should the test focus on Tier 1 capital, Tier 1 Common capital, tangible common equity, a combination of these or something else?
\4 For example, should preferred equity, goodwill and intangible assets be included in the equity component?
\4 Should the denominator be based on risk-weighted assets, tangible assets or something else?
\3 Over what time frame is the stress test examining capital adequacy?
\3 Under what scenarios should regulators apply an other than temporary impairment charge to securities held in a firm's available-for-sale and held-to-maturity portfolios?
\2 Should the stress test measure in-the-moment or measure ``in the stress''?
\2 What economic scenarios are used to stress the firms?
\3 How many economic scenarios are used to stress the firms?
\3 How do these scenarios compare with contemporary private forecasts?
\2 Does the existence or lack of a public capital backstop affect market views of the test's credibility? Without a capital backstop, markets may expect the test to be too optimistic, thereby diminishing the intended effect of the test.
\2 SCAP: Regulators targeted Tier 1 Common Capital and Tier 1 Capital Ratios, examining adequacy through the end of 2010 (approximately 18 months after the results were released). The SCAP used two economic scenarios -- baseline and more adverse -- where the baseline generally matched private forecasts. Moreover the existence of the CAP gave regulators the correct incentives in the test, so that should the test show a firm inadequately capitalized, there was an established source of public capital.
\1 How should a public capital backstop be structured?
\2 What is the basis of this authority?
\2 What sort of preferred security should the public capital be provided through? Should that security be automatically convert to common equity if not redeemed or converted?
\2 Should economic conditions worsen, can the public capital convert into common equity (at a discount)?
\2 What other constraints will firms using public capital face? (E.g. executive compensation caps, restrictions on common stock dividends, buybacks and cash acquisitions)
\2 SCAP: The public backstop, via the CAP, allowed the Treasury to provide additional capital via convertible preferred securities with a 9\% dividend yield. After 7 years the security would automatically convert to common equity if not redeemed or converted before that date. The instrument was designed to give banks the incentive to replace the government-provided capital with private capital as quickly as possible.

\end{outline}

\subsection{Implementation Steps}

\begin{enumerate}

\item Develop the description of the program, including legal authority, purpose, included firms, a general timeline, et cetera and seek input from industry and other stakeholders.
\item If data at a sufficiently granular level is not available from traditional examination processes, collect the relevant data from firms. After providing macroeconomic scenarios to firms, collect estimates for line item losses from each firm. The process of the agency independently producing its own loss estimates and reconciling these with each firms' estimates will likely be the most time intensive aspect of the stress test.
\item If a capital backstop will be used, establish a description of the backstop with legal authority, firms eligible to receive the backstop, the mechanics of the capital injection, et cetera and seek input from industry and other stakeholders.
\item If necessary, seek approval for the program, funding et cetera.
\item Draft detailed FAQs\footnote{SCAP FAQ Example:\url{http://www.federalreserve.gov/newsevents/press/bcreg/bcreg20090225a1.pdf}} and Design and Implementation factsheet.\footnote{SCAP Design and Implementation Example:\url{http://www.federalreserve.gov/bankinforeg/bcreg20090424a1.pdf}}
\item If necessary, develop instructions for completing the documentation necessary to participate in the capital back stop.
\item Publicly release stress test results.

\end{enumerate}

\section{Interview Questions}

\begin{enumerate}

\item
  What was the justification for using the Tier 1 Common capital ratio
  in lieu of the tangible common equity ratio?
\item
  Why was the Federal Reserve chosen as the main regulator to run the
  stress test, and to what extent was Treasury involved between the
  time of the stress test's announcement in February 2009 and the
  results' announcement in May 2009?
\item
  What was the rationale of having the mandatory participation in the
  SCAP set at an asset size of \$100 billion or over? What effect did
  the policymakers and regulators predict this set threshold would have
  on the on-going and future risk management of financial institutions?
\item
  How effective was the SCAP in restoring confidence in the supported
  institutions, the other non-supported institutions and the overall
  financial market? Would the SCAP be effective in preventing and
  managing future crises in not only selected large and complex
  financial institutions, but also the financial markets more broadly?
\item
  Is the fact that CAP went unutilized a sign of an effective government
  effort to restore confidence in the market or a reluctance on the part
  of the banks to take on the government intervention in fear of high
  costs or stigma?
\item
  How did supervisors decide upon capital adequacy ratios of a Tier 1
  capital ratio of six percent and a Tier 1 Common capital ratio of four
  percent?
\item
  Compared to later European stress tests, to what extent was the SCAP
  accepted as credible because of the tough loss estimates and because
  of the credibility of the testing regulators (the Federal Reserve)?
\item
  What was the advantage to measuring the stress rather than
  in-the-moment capital in the Supervisory Capital Assessment Program?
\item
  Why was the window for firms to raise private capital or else turn to
  the CAP six months?
\item
  Discuss alternative ways the regulators could have provided
  transparency in the test.
\item
  How did regulators decide to allow firms with which would receive
  Treasury capital from the CAP to convert the Treasury's preferred
  shares to common equity at a ten percent discount to the firm's stock
  price?
\item
  To what extent did the firms challenge the SCAP's initial findings and
  did this have any impact on the finalized, published stress test
  results?
\item
  What designs of a capital injection program help strike the right
  balance between sufficiently protecting the taxpayer's interests and
  effectively restoring lending activity in the market?
\end{enumerate}


\end{document}
